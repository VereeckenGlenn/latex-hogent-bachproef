%%=============================================================================
%% Inleiding
%%=============================================================================

\chapter{\IfLanguageName{dutch}{Inleiding}{Introduction}}%
\label{ch:inleiding}

De digitalisering van de maatschappij is de afgelopen jaren sterk toegenomen, waardoor bedrijven steeds vaker een digitale voetafdruk nodig hebben \autocite{schwab2016fourth}. \textcite{almeida2020challenges} stellen dat er sinds de coronapandemie een duidelijke drang naar of zelfs eis tot digitalisering is ontstaan. Bij deze digitalisering doen zich echter twee problemen voor: een gebrek aan IT-kennis binnen het bedrijfsleven \autocite{kutnjak2021covid} en een tekort aan ICT-professionals of de mogelijkheid om deze aan te nemen \autocite{VDAB2024}. 

\vspace{\baselineskip}

De covid-pandemie dwong bedrijven zich snel op de digitale markt te positioneren vanwege de plotselinge overgang naar lockdown. Jongere generaties raakten tijdens de pandemie vertrouwd met webshops \autocite{almeida2020challenges}, waardoor het voor achterblijvende bedrijven noodzakelijk werd om snel en eenvoudig te digitaliseren. Volgens \textcite{findikoglu2011small} verkiezen kleine tot middelgrote bedrijven het werken met locale service-providers, aangezien het makkelijker is om een vertrouwensband te vormen op lokale basis. Low-code kan een mogelijke oplossing bieden voor de efficiënte ontwikkeling van websites en applicaties zodat kleinere, locale IT-bedrijven meer contracten kunnen aannemen. 

\vspace{\baselineskip}

Volgens \textcite{bock2021low} is Low-code (LC) een relatief nieuwe programmeermethode die \\sinds het einde van de jaren 2010 snel aan populariteit wint. hij vermeldt verder dat low-code platforms (LCP’s) een grafische gebruikersinterface (GUI) bieden waarmee ontwikkelaars applicaties kunnen bouwen met minimale hoeveelheden code. Als laatste merkt hij op dat het aantal bedrijven dat LCP's aanbiedt elk jaar groeit en zowel kleine bedrijven als tech-giganten zoals IBM, Microsoft en Oracle omvat.

\section{\IfLanguageName{dutch}{Probleemstelling}{Problem Statement}}%
\label{sec:probleemstelling}

%TODO

Uit je probleemstelling moet duidelijk zijn dat je onderzoek een meerwaarde heeft voor een concrete doelgroep. De doelgroep moet goed gedefinieerd en afgelijnd zijn. Doelgroepen als ``bedrijven,'' ``KMO's'', systeembeheerders, enz.~zijn nog te vaag. Als je een lijstje kan maken van de personen/organisaties die een meerwaarde zullen vinden in deze bachelorproef (dit is eigenlijk je steekproefkader), dan is dat een indicatie dat de doelgroep goed gedefinieerd is. Dit kan een enkel bedrijf zijn of zelfs één persoon (je co-promotor/opdrachtgever).

\section{\IfLanguageName{dutch}{Onderzoeksvraag}{Research question}}%
\label{sec:onderzoeksvraag}

Gezien de sterk evoluerende trend naar digitalisering, is het interessant om te onderzoeken hoe kleine tot middelgrote bedrijven zich kunnen digitaliseren. De focus ligt hierbij op het ontwikkelen van een persoonlijke website met mogelijke integratie van een webshop. De onderzoeksvraag luidt als volgt: 
 
\vspace{\baselineskip}

\begin{itemize}
  \item Welk low-code ontwikkelingsplatform voldoet het best aan de eisen voor de ontwikkeling van een moderne website voor kleine tot middelgrote bedrijven?
\end{itemize}

\subsubsection{Subvragen}
\label{sec:Subvragen}
\begin{itemize}
  \item Welk low-code ontwikkelingsplatform is de meest betaalbare optie voor kleine bedrijven?
  \item Welk low-code ontwikkelingsplatform biedt de meeste functionaliteiten en customisatie aan de gebruiker?
  \item Welk low-code ontwikkelingsplatform is het gemakkelijkst te gebruiken voor \\IT-personeel?
\end{itemize}

\section{\IfLanguageName{dutch}{Onderzoeksdoelstelling}{Research objective}}%
\label{sec:onderzoeksdoelstelling}

Wat is het beoogde resultaat van je bachelorproef? Wat zijn de criteria voor succes? Beschrijf die zo concreet mogelijk. Gaat het bv.\ om een proof-of-concept, een prototype, een verslag met aanbevelingen, een vergelijkende studie, enz.

\section{\IfLanguageName{dutch}{Opzet van deze bachelorproef}{Structure of this bachelor thesis}}%
\label{sec:opzet-bachelorproef}

% Het is gebruikelijk aan het einde van de inleiding een overzicht te
% geven van de opbouw van de rest van de tekst. Deze sectie bevat al een aanzet
% die je kan aanvullen/aanpassen in functie van je eigen tekst.

De rest van deze bachelorproef is als volgt opgebouwd:

\begin{itemize}
  \item \hyperref[sec:literatuurstudie]{Hoofdstuk 2} bespreekt de literatuurstudie en de stand van zaken rondom low-code. Dit onderdeel verduidelijkt wat low-code en low-code platformen precies zijn. Verder wordt er gekeken naar welke componenten zorgen voor een goede en moderne website. Tenslotte wordt er nog gekeken naar voorgaand onderzoek en onderzoek in verband met de leerbaarheid van low-code platformen.
  \item \hyperref[sec:methodologie]{Hoofdstuk 3} geeft uitleg over de methodologie die gebruikt zal worden tijdens dit onderzoek. Eerst wordt de requirements-analyse besproken. Vervolgens worden de longlist en de shortlist besproken. Daarna wordt de proof of concept van dit onderzoek besproken. Tenslotte wordt er nog gesproken over de verwerking van de data die uit de proof of concept en de conclusie die daaruit voortvloeien.
  \item \hyperref[sec:Verwachte resultaten]{Hoofdstuk 4} bespreekt de verwachte resultaten die verzameld zijn op basis van de literatuurstudie en de methodologie. Dit schept een duidelijk beeld van de informatie die met dit onderzoek is verkregen. 
  \item \hyperref[sec:discussie-conclusie]{Hoofdstuk 5} bespreekt de verwachte conclusies uit het onderzoek. In deze conclusie wordt een antwoord gegeven op de onderzoeksvraag en de bijbehorende subvragen.
\end{itemize}