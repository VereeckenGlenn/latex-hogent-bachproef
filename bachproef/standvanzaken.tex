\chapter{\IfLanguageName{dutch}{Stand van zaken}{State of the art}}%
\label{ch:stand-van-zaken}

% Tip: Begin elk hoofdstuk met een paragraaf inleiding die beschrijft hoe
% dit hoofdstuk past binnen het geheel van de bachelorproef. Geef in het
% bijzonder aan wat de link is met het vorige en volgende hoofdstuk.

% Pas na deze inleidende paragraaf komt de eerste sectiehoofding.

n de eerste hoofdstukken van deze literatuurstudie worden belangrijke termen gedefinieerd. Vervolgens worden de verschillen tussen low-code en no-code platforms verkend, waarbij wordt besproken hoe beide werken en voor welk publiek ze bedoeld zijn. Daarna komt de opzet van een moderne website aan bod, met een focus op welke elementen bijdragen aan het succes ervan. Ook wordt eerder onderzoek naar low-code onderzocht om een beter inzicht te krijgen in de voordelen en beperkingen die in eerdere studies zijn vastgesteld.

\subsection{Definities}
\label{sec:Definities}

\textbf{Low-code}: low-code is een meer visuele manier van softwareontwikkeling waarbij ontwikkelaars gebruik kunnen maken van een grafische interface die reeds bestaande codefragmenten gebruikt om te programmeren. In tegenstelling tot pure code moet de programmeur veel minder code zelf schrijven. Bij low-code wordt er wel nog vanuit gegaan dat de programmeur zelf nog onderdelen van software codeert dat niet kan gedaan worden aan de hand van het low-code platform. \autocite{Spain2022}

\vspace{\baselineskip}

\textbf{No-code}: no-code is een manier van softwareontwikkeling waarbij de gebruiker zelf geen code hoeft te schrijven. Het interessante aan no-code platformen is dat er weinig tot geen programmeerkennis voor nodig is om deze te kunnen gebruiken. Met behulp van het no-code platform kan de gebruiker zelf zijn volledige applicatie ontwikkelen in een grafische interface zonder te hoeven weten hoe de achterliggende processen werken. \autocite{Kirvan2024}

\vspace{\baselineskip}

\textbf{Citizen developer}: Een citizen developer is een medewerker binnen een organisatie zonder formele opleiding of achtergrond in softwareontwikkeling, die zich bezighoudt met het ontwikkelen en onderhouden van softwareapplicaties. Dit gebeurt meestal via een low-code of no-code ontwikkelingsplatform, waarmee deze ontwikkelaar op eenvoudige en gebruiksvriendelijke manier aan softwareontwikkeling kan doen zonder daarvoor kennis nodig te hebben van het programmeren. Vaak zijn citizen developers afkomstig uit de zakelijke kant van het bedrijf. \autocite{Kirvan2023}

\subsection{Low-code vs No-code}
\label{sec:Low-code vs No-code}

Volgens \textcite{Oluwaseyi2024} is het grootste verschil tussen deze twee programmeerstijlen voornamelijk de doelgroep. Zowel low-code als no-code zijn beide manieren om software te ontwikkelen op basis van een GUI met vooraf gedefinieerde componenten. Hij beweert dat bij low-code het platform hoofdzakelijk gericht is op programmeurs die ervaring hebben met programmeren en begrijpen wat er achterliggend allemaal gebeurt bij een applicatie. Vervolgens zegt \textcite{Oluwaseyi2024} dat bij no-code het product meer gericht is op zogenaamde “citizen developers” of mensen uit de business die geen voorafgaande kennis hebben van programmeren. Hierbij wordt het volledige ontwikkelingsproces zo veel mogelijk vereenvoudigd tot simpele componenten waarvan de code niet kan worden gewijzigd. Hij beweert ook dat low-code daarentegen de gebruiker meer de mogelijkheid zal geven om componenten aan te passen of te wijzigen. Met deze ondervindingen duidt \textcite{Oluwaseyi2024} een zeker verschil aan in de flexibiliteit en manier van werken tussen beide werkwijzen.

\vspace{\baselineskip}

Volgens \textcite{ICE2022} is er tenslotte ook nog een verschil in doel. \textcite{ICE2022} beweert dat het hoofddoel van low-code is om vaste ontwikkelingsprocessen te versnellen door de programmeur te voorzien van flexibele codeblokken, zodat er niet te lang stilgestaan moet worden bij deze onderdelen van het ontwikkelingsproces. \textcite{ICE2022} zegt ook dat bij no-code het vooral de bedoeling is om mensen met weinig tot geen programmeerkennis de mogelijkheid te geven om hun eigen applicaties te bouwen op een gebruiksvriendelijke manier.

\subsection{Layout van een moderne website}
\label{sec:Layout van een moderne website}

Hoewel webdesign niet de kern van dit onderzoeksvoorstel is, is het van belang kort te bespreken welke factoren een website succesvol maken. User engagement, user satisfaction en gebruiksvriendelijkheid zijn cruciale elementen bij het ontwikkelen van een website. User satisfaction en gebruiksvriendelijkheid kunnen samengevoegd worden om aan te geven dat het vooral gaat om het gebruiksgemak. Onderzoek van onder andere \textcite{vila2021indicators}, \textcite{saoula2023building} en \textcite{flavian2009web} ondersteunt dit.

\vspace{\baselineskip}

User engagement verwijst naar het daadwerkelijk gebruiken van een website door gebruikers, wat bij een webshop ook betrekking heeft op het stimuleren van verkoop. \textcite{garett2016literature} identificeren navigeerbaarheid, visualisatie en lay-out als de belangrijkste elementen van user engagement.

\vspace{\baselineskip}

Navigeerbaarheid betreft de mogelijkheid voor de gebruiker om zich binnen een website te navigeren. Volgens onderzoek van \textcite{vila2021indicators} begint 70\% van de gebruikers met zoeken zonder precies te weten wat en hoe ze willen zoeken. Daarom is de bruikbaarheid en herkenbaarheid van navigatie componenten essentieel om frustratie bij gebruikers te voorkomen.

\vspace{\baselineskip}

Visualisatie slaat op het visueel en ordelijk representeren van data en bijvoorbeeld ook producten in een webshop. Het principe van visual marketing stelt dat klanten eerder geneigd zijn iets te kopen als ze kunnen zien hoe een product eruitziet. \textcite{saoula2023building} tonen aan dat een visueel en ordelijk gepresenteerde website het vertrouwen van gebruikers vergroot, wat hen stimuleert om terug te keren.

\vspace{\baselineskip}

Lay-out is belangrijk in twee opzichten: kleurgebruik en de presentatie van verschillende elementen van een website. Volgens \textcite{beaird2020principles} is het belangrijkste aan een lay-out dat een website voelt als een geheel. Verder vindt hij dat een goede mix aan content en zogenaamde "whitespace" essentieel is bij het ontwerpen van een goede lay-out.

\vspace{\baselineskip}

Responsiviteit voor verschillende apparaten is eveneens cruciaal. \textcite{wagner2020online} constateerden dat in 2017 wereldwijd 49\% van de online shoppers gebruikmaakte van hun pc, terwijl 51\% gebruikmaakte van mobiele apparaten zoals smartphones en tablets. Hedendaags is dit percentage voor de mobiele markt gestegen tot meer dan 60\% van de totale online shoppers \autocite{Seitz2024}.

\subsection{Voorgaand onderzoek}
\label{sec:Voorgaand onderzoek}

In de laatste jaren is het gebruik van low-code en het onderzoek hiernaar sterk in opmars. Steeds meer applicaties worden ontwikkeld met behulp van low-code. \textcite{GartnerResearch2019} voorspelde in 2019 dat tegen 2024 meer dan 65\% van alle ontwikkelingsprojecten gebruik zullen maken van \\low-code voor de ontwikkeling van hun applicaties. Een onderzoek van \textcite{VanMullem2022} onderzoekt of traditioneel programmeren volledig kan worden vervangen door low-code. Hij concludeert dat ongeveer 95\% van processen kunnen worden gemodelleerd met behulp van een LCP, de overige 5\% bevat taken die te complex waren voor low-code platformen.

\subsection{De bruikbaarheid van low-code}
\label{sec:De bruikbaarheid van low-code}

Volgens het boek van \textcite{kenneweg2021building} kan elke geïnteresseerde persoon applicaties maken met behulp van een low-code platform. Voorafgaande kennis in softwareontwikkeling is niet noodzakelijk om met low-code applicaties te werken volgens hen, hoewel het wel een pluspunt kan zijn om het leer- en werkproces te vereenvoudigen.

\vspace{\baselineskip}

\textcite{rokis2022challenges} stellen dat low-code technologieën veel uitdagingen met zich meebrengen. Volgens hen zouden deze platforms een gebrek hebben aan methoden en benaderingen ter ondersteuning. Dit probleem kan volgens hen worden verminderd door de documentatie en leermiddelen van low-code platforms te verbeteren. \textcite{bernsteiner2022citizen} merken op dat de huidige documentatie vaak geschreven is voor professionele developers, wat het aanleren van deze software makkelijker maakt voor actieve ontwikkelaars.